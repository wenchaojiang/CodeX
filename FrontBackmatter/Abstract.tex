%*******************************************************
% Abstract
%*******************************************************
%\renewcommand{\abstractname}{Abstract}
\pdfbookmark[1]{Abstract}{Abstract}
\begingroup
\let\clearpage\relax
\let\cleardoublepage\relax
\let\cleardoublepage\relax

\chapter*{Abstract}
\noindent This thesis contributes to the understanding of the potential socio-technical issues that might emerge from the interaction between responder teams and automated planning support and propose design solutions to them. \\

(Problem) Recently, Frequent natural and man-made disasters in Haiti, Chile and Japan drew attention of Researchers. A lot efforts have been made to study the technologies that can assist human responders to improve their performance. In the disaster response domain,  a disaster response team, which contains several incident commanders and field agents, is faced with the problem of carrying out geographically distributed tasks under spatial and time constraints in a quickly changing task environment. \\ 

\noindent Effective planning and coordination can be a key factor for the success of disaster operation but it is difficult to achieve. Recent advance in the multi-agent technologies leads to the possibility of building agent software which supports Team coordination by automating the task planning process. However , it is unknown how the agent software can fit into the team organisation in a way that improve rather than hinders the team performance. The interaction between human operators and planning support systems need to be carefully designed before technology deployment.\\

\noindent (method) This work presents three field studies which investigates the impact of different interactional arrangements between human teams and automated planning support. The studies adopt serious game approach which is arguably an established vehicle to vehicle to explore socio-technical issues in complex real world settings.\\

\noindent  We developed AtomicOrchid, an emergency response game to create a task setting which mirrors real aspects of disaster response operation. In the game trials, participants are recruited to play as field responders and incident commanders to carry out rescue missions. Participants' experiences are observed and recorded as they coordinate with each other to achieve game objectives, with the support from an intelligent planner software. Interaction analysis is carried out on the data, leading to descriptive results which unpacks interactional issues. By iteratively designing and examining different interactional arrangements through three iteration of studies, we progressively explore requirements and social implications of planning support system for responder teams.\\

\noindent In the 1st study,field responders and incident commander coordinate without support of the intelligent planner. The study establish baseline performance of the game play and derived several requirements for planning support system. In the 2ed study, an intelligent planner was introduced to support field responders directly. In the third study, Incident commander mediate task assignment between field responders and the planning agent. \\

\noindent (results) Overall these studies show that ... 


\endgroup			

\vfill