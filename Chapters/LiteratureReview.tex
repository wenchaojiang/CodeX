%*****************************************
\chapter{Literature Review}\label{ch:examples}
%*****************************************
\section{Automation and Human Agent interaction}
The study Automation design and Human agent interaction are two heavily intersected research areas which overarches this PhD work.  


\subsection{ A History of Automation }
The original goal of Automation is to replace the tasks originally performed by human with a machine. \cite{Bradshaw2011} . It can be defined as the execution by machine, usually computer, of a function previously performed by human [Parasuraman and Riley 1997] The tasks that can be automated is used to be limited by technical capabilities, but this is no longer the case. With quick growth of machine's speed and intelligence, the tasks that can be automated is rapidly increasing, including complex cognitive activities such as information analysis, planning and decision making.[RAJA] The boundary between human machine capabilities has blurred. With little cannot be automated, the automation designers have to make hard choice about what to automate and to what extent.\\

One traditional approach for automation design is to simply automate all system functions that can be automated easily in a cost-effective way, leaving the all remaining tasks to human operators. The main considerations in this approach are technical capability and  cost. The assumption of this approach is that the automation of sub systems functions can lead to optimisation of whole system with no detrimental impact results from the automation. However this is not always the case, Large body of empirical work in automation design have shown that xxxx .\\ 

The other approach is to achieve division of labour between human and automation according to their strength and weakness. As in Fitts list \cite{Fitts} , a set of strengths and weaknesses of humans and machines is identified. However the division of labour would not be as simple as a labour division according to strength and weakness. Firstly the time factor could be important because human and machine's availability may change overtime. Secondly By delegating the same task to machine, the nature of human tasks can be changed as well. Large body of work has shown clearly that automation does not simply supplant human activity but rather changes it, often in a way that is unanticipated by the system designer. [Coordination and Supervision interaction required, expand ] \cite{Bradshaw2011} [Later , proposed by Licklider in the un-Fitts list \cite{Hoffman2002}, automation is aimed to leverage and extent human capability by using agents (not replacing).\cite{Bradshaw2011}.] \\


\subsection{The Level of Automation}
More recently, another alternative model of automation to guide automation design is proposed by proposed by xxx,known as 10 level model of automation. The model 10 levels of automation (LOA) has been later extended by xxx with more detailed classification of automation types. The extended model of LOA informs study approach of this PhD work, which will be documented in chapter x. This section will give an introduction of this model. 


Most tasks can be fully or partially automated, which implies that automation is not all or none, but can vary across a continuum of level[]. At the lowest level, all system functions are performed manually by human operators.  At highest level, system are fully automated, taking over all system functions. In between this two extremes, 10 different levels of automation are proposed by xxx. As shown in table x. 


\subsection{ Definition of Software Agent }

Since creation of the term "software agent", there are a lot of debates about its definition . One definition commonly shared by the Multi-Agent system(MAS) literatures is that the software agents are designed to operate independently without constant human supervision. In the AI community, the software agent evolves from multi-agent system research which in turn, derived from the field of Distributed AI \cite{Vlassis2007}. This strand of work investigates infrastructure, language and communication to realize coordinated agent software system. The goal was to specify, analyse, design and integrate systems comprising of multiple collaborative agents.\cite{Nwana1996} \\

More recently, the use of the word software agent become much more diversified. \cite{Nwana1996} has made attempts to investigate broader classes and types of agent. Nwana's \cite{Nwana1996} topology of agents identified three characteristics of agents, learning, autonmous and collabroitive, Based on this characteristics, software agents can be categorised into 8 classes, ranging from collaboration agents, information agents to interface agents. \\

With a broader definition, researchers begin to investigate broader issues surrounding development of agent systems, among which human agent interactions have attracted a lot of attention of researchers and practitioners.[get one reference here!!!]

\subsection{ Human Agent Interactions }

Automation design and Human agent interactions have big overlaps as both concerns the impact of automation(by computational systems) on human performance. While automation design is trying to answer the question of what and to what extent the automation should be, the study of human agent interaction concerns the issue of interaction design related to development of human agent system, with the aim to develop agents which is good at teamworking with human operators. 

The agent researchers adopted a view that human and agents are equally important team players in a human agent system.[] It highlights the importance of mutual interaction between human and agents that can enhance the competencies of both human and systems. From this perspective, the aim of automation is no longer "replace" human but achieve an human-machine symbiosis through effective human agent coordination, which is similar to the view adopted by xxx in developing un-fitts list. As researchers realize the effective human-machine symbiosis requires sophisticated interactions design between human and agent, the "interactional" issues of automation is thought to be as important as technical issues.[Expand discussion about this issues.] \cite{Bradshaw2011} \\

Several interaction techniques have been devised by agent researchers to achieve effective coordination between human and agent. 

[define the interactional issues!!!!]




\subsection{ Issues in Human Agent Interactions }
Fusion of buothon 





Why the social issues of designing agent system is important? [Norman1994]

Flexible autonomy.

Mixed initiative.

\subsection{lesson from CSCW}

\subsection{A summary}

Because this PhD work sits in the overlapping research area of human agent interaction and automation design, the later chapters will consider automation and agent software as two interchangeable concepts.


\section{Task Planning in Disaster Response}\label{ch:teams}

What is coordination(Coordination theory) Team Coordination. Malone (1990 361) defines coordination as the act of managing interdependencies between activities performed to achieve a goal. One of the very important component of coordination in DR, following sections will firstly... and then discuss how the coordination is carried out through command structure of DR team. 

\subsection{Task Planning}
In disaster response, team coordination is essential in order that groups of people can carry out interdependent activities together in a timely and satisfactory manner (cf. Bradshaw et al., 2011). Disaster response experts report that failures in team coordination are the most significant factor in critical emergency re- sponse (Toups et al., 2011: 2) that can cost human lives. Shared understanding, situation awareness, and alignment of cooperative action through on-going communication are key requirements to enable successful coordination. Convertino et al. (2011) design and study a set of tools to support common ground and aware- ness in emergency management. \\

 One important characteristic of large-scale disaster is the presence of multiple spatially distributed incidents (Chen et al., 2005). To deal with multiple incidents, the disaster response team has to coordinate spatially distributed resources and personnel to carry out operations (e.g. search, rescue and evacuation). Depending on the proliferation of incidents, response personnel may need to dispatch, deploy and redeploy limited resources. Coordination is required to efficiently allocate limited resources to multiple incidents with temporal and spatial constraints imposed by the nature of disasters.\\

(The requirements of multiple incident response from Chen.)


((Coordination game) Below confirms that in a single target need multiple responder teams)
While the present work applies generally to disaster response, our iterative design and theory-building processes have been specifically informed by work practice in the subdomain of fire emergency response. We work from fire emergency response in small-scale structural fires, observing practice in the United States of America. Fire emergency response is undertaken by small teams distributed throughout the incident, coordinated by an incident commander (IC) [Toups and Kerne 2007; Landgren 2006; Jiang et al. 2004; Landgren and Nulden 2007; Wieder et al. 1993; Carlson 1983; U.S. Department of Homeland Security 2008]. Multiple response teams, or companies, are dispatched to any incident and cooperate around the fireground (Figure 1). A company officer leads each team, which consists of firefighters and/or engineers.2 Normally, each company is associated with a firefighting vehicle; an apparatus, such as an ambulance, engine, or ladder truck.\\

The fireground and surrounding space constitute a dangerous and dynamic interface ecosystem [Kerne 2005] of distributed cognition, connecting responders, victims, fire- fighting equipment, communication media, and information artifacts. Upon arriving at an incident, multiple companies distribute in and around the fireground. Compa- nies and their apparatuses are placed at strategic locations, and are moved as needed. Human operators work on and from these platforms. Firefighters and rescue workers deploy from them, taking equipment into the fireground; equipment, such as firehoses and radios, may be technologically supported by the apparatus itself (pumps and water sources, or high-power repeaters, respectively). Each apparatus, and in many cases, each human worker, is equipped with a half-duplex radio to facilitate long-range, broad- cast communication.\\

(Geographic distribution as a also a key factor)

\noindent Muti-objective natural of the the DR response lead to its connection with multi-agent simulation and optimisation. 

Therefore task planning envlove xxx challenges and opens the opportuntiy for technology intervention. Current state of art practice will be reviewed in section........\\

\subsection{DR Command structure}


A relatively generic model of C and C is proposed by 
The development of Information Technology System (ITS)
to support the emergency response team work has many work published in the literature. For long time, crisis management systems were based in the military model of CC [1]. For many authors [2, 3 and 4] the need to change the theories of emergency management creating new paradigms is imperative to improve the flexibility of the CC structures. Their aim is to make them more
efficient, multi-disciplinary and multi-
institutional, increasing the collaboration between CC and the field responders and allowing sharing planning and resources to stabilize the crisis [5].
Stanton et. Al. [7] proposes a new generic model of CC
based on field observation. One important conclusion he made concerning common characteristics in the different domains of CC:
1. The presence of a remote control room; 2. The great dependency on verbal communication and;
3. The existence of collaborative discussion between field teams and command.\\


Gold-Sliver-Bronze control model is documented in xxxx(wiki)

\section{Technology support for task planning}
This section reviews state of art technological practice to support task planning. Task planning toolbox containing various components including data sets, (meta-)information, storage and query tools, analysis methods, theories, indicators.

Greetman review of planning system. not real time, The systems can be categorized according to several criteria (Aims, Capabilities, Content, Structure, Technology). None of them is adapted to real-time coordination.

Multi-agent paraghim 


Graph to position the my system. 

\subsection{Disaster simulation, optimisation}
The section will briefly review current status of the simulation and optimisation technologies for disaster response.



\section{A Ecosystem view}

Taking Muti-agent paradim and ecological view of disaster response operation. Combined them with DR response structure to produce a view of future Dr ecological system backed by simulation and optimisation technologies. 

So what is the problem now, -> social ! issue. 





\subsection{Workflow Management}



\section{Serious Game}








%*****************************************
%*****************************************
%*****************************************
%*****************************************
%*****************************************
