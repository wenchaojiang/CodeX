%*****************************************
\chapter{Literature Review}\label{ch:examples}
%*****************************************
\section{Software Agent and Human Agent interaction}



\subsection{ A History of Automation }
The definition of Automation is to replace the tasks originally performed by human with a machine. \cite{Bradshaw2011} . At early stages, the straightforward question is that what task can automated?  Fitts listed strengths and weaknesses of humans and machines to identify what can be automated. \cite{Fitts} . \\

However the division of labour would not be as simple as labour division according to strength and weakness. Firstly the time factor could be important because human and machine's availability may change overtime. Secondly By delegating the same task to machine, the nature of human tasks can be changed as well. \cite{Bradshaw2011} \\

Movevoer,w ith advance of technologies, the boundary between human machine capabilities has blurred. Machine can involved in sophisticated judgements now-days. A new approach to view the human machine relationship has been proposed by Licklider called un-Fitts list \cite{Hoffman2002} in which automation is aimed to leverage and extent human capability by using machines.\cite{Bradshaw2011}. The un-fitts list highlights the importance of mutual interaction between human and agents that can enhance the competencies of both human and agents. From this perspective, the aim of automation is no longer "replace" human but achieve an effective human-machine symbiosis. As researchers realize the effective human-machine symbiosis requires sophisticated social interactions between human and agent, the "social" issues of automation is thought to be as important as technical issues. \cite{Bradshaw2011} \\


\subsection{ Definition of Software Agent }

Since creation of the term "software agent", there are a lot of debates about its definition . One definition commonly shared by the Multi-Agent system(MAS) literatures is that the software agents are designed to operate independently without constant human supervision. In the AI community, the software agent evolves from multi-agent system research which in turn, derived from the field of Distributed AI \cite{Vlassis2007}. This strand of work investigates infrastructure, language and communication to realize coordinated agent software system. The goal was to specify, analyse, design and integrate systems comprising of multiple collaborative agents.\cite{Nwana1996} \\

More recently, the use of the word software agent become much more diversified. \cite{Nwana1996} has made attempts to investigate broader classes and types of agent. Nwana's \cite{Nwana1996} topology of agents identified three characteristics of agents, learning, autonmous and collabroitive, Based on this characteristics, software agents can be categorised into 8 classes, ranging from collaboration agents, information agents to interface agents. \\

The thesis will adopt a boarder view of the agent software. \\

\subsection{ Issues in Human Agent Interactions }

Why the social issues of designing agent system is important? [Norman1994]

Flexible autonomy.

Mixed initiative.

\section{Task Planning in Disaster Response}\label{ch:teams}

What is coordination(Coordination theory) Team Coordination. Malone (1990 361) defines coordination as the act of managing interdependencies between activities performed to achieve a goal. One of the very important component of coordination in DR, following sections will firstly... and then discuss how the coordination is carried out through command structure of DR team. 

\subsection{Task Planning}
In disaster response, team coordination is essential in order that groups of people can carry out interdependent activities together in a timely and satisfactory manner (cf. Bradshaw et al., 2011). Disaster response experts report that failures in team coordination are the most significant factor in critical emergency re- sponse (Toups et al., 2011: 2) that can cost human lives. Shared understanding, situation awareness, and alignment of cooperative action through on-going communication are key requirements to enable successful coordination. Convertino et al. (2011) design and study a set of tools to support common ground and aware- ness in emergency management. \\

 One important characteristic of large-scale disaster is the presence of multiple spatially distributed incidents (Chen et al., 2005). To deal with multiple incidents, the disaster response team has to coordinate spatially distributed resources and personnel to carry out operations (e.g. search, rescue and evacuation). Depending on the proliferation of incidents, response personnel may need to dispatch, deploy and redeploy limited resources. Coordination is required to efficiently allocate limited resources to multiple incidents with temporal and spatial constraints imposed by the nature of disasters.\\

(The requirements of multiple incident response from Chen.)


((Coordination game) Below confirms that in a single target need multiple responder teams)
While the present work applies generally to disaster response, our iterative design and theory-building processes have been specifically informed by work practice in the subdomain of fire emergency response. We work from fire emergency response in small-scale structural fires, observing practice in the United States of America. Fire emergency response is undertaken by small teams distributed throughout the incident, coordinated by an incident commander (IC) [Toups and Kerne 2007; Landgren 2006; Jiang et al. 2004; Landgren and Nulden 2007; Wieder et al. 1993; Carlson 1983; U.S. Department of Homeland Security 2008]. Multiple response teams, or companies, are dispatched to any incident and cooperate around the fireground (Figure 1). A company officer leads each team, which consists of firefighters and/or engineers.2 Normally, each company is associated with a firefighting vehicle; an apparatus, such as an ambulance, engine, or ladder truck.\\

The fireground and surrounding space constitute a dangerous and dynamic interface ecosystem [Kerne 2005] of distributed cognition, connecting responders, victims, fire- fighting equipment, communication media, and information artifacts. Upon arriving at an incident, multiple companies distribute in and around the fireground. Compa- nies and their apparatuses are placed at strategic locations, and are moved as needed. Human operators work on and from these platforms. Firefighters and rescue workers deploy from them, taking equipment into the fireground; equipment, such as firehoses and radios, may be technologically supported by the apparatus itself (pumps and water sources, or high-power repeaters, respectively). Each apparatus, and in many cases, each human worker, is equipped with a half-duplex radio to facilitate long-range, broad- cast communication.\\

(Geographic distribution as a also a key factor)

\noindent Muti-objective natural of the the DR response lead to its connection with multi-agent simulation and optimisation. 

Therefore task planning envlove xxx challenges and opens the opportuntiy for technology intervention. Current state of art practice will be reviewed in section........\\

\subsection{DR Command structure}


A relatively generic model of C and C is proposed by 
The development of Information Technology System (ITS)
to support the emergency response team work has many work published in the literature. For long time, crisis management systems were based in the military model of CC [1]. For many authors [2, 3 and 4] the need to change the theories of emergency management creating new paradigms is imperative to improve the flexibility of the CC structures. Their aim is to make them more
efficient, multi-disciplinary and multi-
institutional, increasing the collaboration between CC and the field responders and allowing sharing planning and resources to stabilize the crisis [5].
Stanton et. Al. [7] proposes a new generic model of CC
based on field observation. One important conclusion he made concerning common characteristics in the different domains of CC:
1. The presence of a remote control room; 2. The great dependency on verbal communication and;
3. The existence of collaborative discussion between field teams and command.\\


Gold-Sliver-Bronze control model is documented in xxxx(wiki)

\section{Technology support for task planning}
This section reviews state of art technological practice to support task planning. Task planning toolbox containing various components including data sets, (meta-)information, storage and query tools, analysis methods, theories, indicators.

Greetman review of planning system. not real time, The systems can be categorized according to several criteria (Aims, Capabilities, Content, Structure, Technology). None of them is adapted to real-time coordination.

Multi-agent paraghim 


Graph to position the my system. 

\subsection{Disaster simulation, optimisation}
The section will briefly review current status of the simulation and optimisation technologies for disaster response.



\section{A Ecosystem view}

Taking Muti-agent paradim and ecological view of disaster response operation. Combined them with DR response structure to produce a view of future Dr ecological system backed by simulation and optimisation technologies. 

So what is the problem now, -> social ! issue. 


\section{CSCW}


\subsection{Workflow Management}



\section{Serious Game}








%*****************************************
%*****************************************
%*****************************************
%*****************************************
%*****************************************
