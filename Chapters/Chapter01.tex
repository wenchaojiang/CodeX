%************************************************
d\chapter{Introduction}\label{ch:introduction}
%************************************************
Coordination in Disaster Response (DR) operations such as Urban Search And Rescue (USAR) can be very challenging. In large-scale disaster, DR teams may have limited resource and personnels to deal with multiple incidents across a large impact area. Task planning and execution need to be carried out by geographically distributed DR teams in real time against uncertainties in the environment. The challenges highlight the opportunity space of technology support for real time task planning and execution.  \\ 

Recently, various Information and Communication Technologies (ICT), ranging from communication infrastructures to social media platforms, have been playing increasingly significant role in the disaster management.  Moreover, Muti-agent system researchers have devised various real-time task planning algorithms to automate planning in time critical task domains such as disaster response. The advances in both ICT and Multi-agent optimisation algorithms lead to the opportunity of intelligent planning support in the DR domain. However, before we apply this algorithm to to support DR operations, we need to understand how can we design the interaction between responder team and the planning support in a way that improve, rather than hinder team performance. Empirical studies of CSCW systems have shown that it is vital to study technology in use to understand potential tensions between social and technical aspects of a system. In particular, field studies of workflow support systems have revealed that technologies can disrupt rather than smooth workflow if they are not designed in a socially acceptable way. The aim of this work is to explore potential socio-technical issues surrounding a planning support system, which in turn, informs interaction design of such systems. \\

This PhD work is a ORCHID sponsored research, which contributes to the understanding needed to build Human Agent Collectives (HACs) in disaster domain. As computational systems  becoming increasingly embedded into our life, the researchers from ORDHID project envision a future in which people and computational agents operates at a global scale, forming human agent collectives. The ORCHID project is aimed to realise the vision of HACs by studying the science that is needed to understand, build and apply HACs that symbiotically interleave human and computer systems (www.orchid.ac.uk).\\

This chapter will give an overview of the PhD work, which covers research objectives, approach, research questions and contributions, followed by a list of publications related to this thesis and an overview of the thesis structure.\\


\section{Problem Definition and Objectives}
In large scale disasters, Disaster Response (DR) team may have limited resource and personnels to deal with large amount of incidents across large geographic area under time pressure. In this situation, task and team allocation become a grand challenge for DR team. The responders and resources need to be assigned to teams and tasks in a way that to minimise loss of life and costs (e.g., time or money) . For instance responders with different capabilities (e.g., fire-fighting or life support) have to form teams in order to perform rescue tasks (e.g., extinguishing a fire or providing first aid). Thus, responders have to plan their paths to the tasks (as these may be distributed in space) and form specific teams to complete them. These teams, in turn, may need to disband and reform in different configurations to complete new tasks, taking into account the status of the current tasks (e.g., health of victims or building fire) and the environment (e.g., if a fire or radioactive cloud is spreading). Furthermore, uncertainty in the environment (e.g.,road connectivity, task status update) or in the responders abilities to complete tasks (e.g., some may be tired or get hurt) means that plans are likely to change continually to reflect the prevailing assessment of the situation.\\

Recent advances in multi-agent systems research leads to a number of real-time simulation and optimization technologies, some of which has great potential to be adapted to support task planning for DR teams (More details, see section x). Although the opportunity space has been recognised, most multi-agent coordination algorithms has only be tested in computational simulations. None of them has be deployed to guide real human in DR situations. However, extreme difficulties might be encountered when introducing new technology support for human teams. New technologies might not support, but may disrupt smooth workflow if they are designed in an organisationally unacceptable way. Many CSCW literatures have pointed out ill-designed work-flow management/automation system can lead to undesirable results, not only fail to improve work efficiency but also hinders human performance. Field studies of CSCW technologies have shown that it is vital to study technology in use to understand potential tensions raised for teamwork.  \\

We believe the same is true for automated planning support. Before we can build automated systems that support human task planning, field trials are required to understand the potential impact of technology support for team coordination. Although most multi-agent coordination algorithms have been tested to perform well in the computational simulations, they have never been excised to guide real human responders in the real world environment. Currently, there are few studies aimed to unpack the interactional issues relating to a socio-technical aspects of the intelligent planning support system. The objective of this PhD work is to fill this gap by exploring and unpacking interactional issues surrounding the intelligent planning support system from a socio-technical perspective.\\

\section{Approach}\label{sec:custom}

To meet our research objective, we adopt a serious mixed reality games approach (chapter \ref{ch:approach}) to create a game probe (i.e. AtomicOrchid) that enables studying team interaction with planning support system in a disaster scenario whilst providing confidence in the efficacy of behavioural observations. Mixed-reality games bridge the physical and the digital. Arguably, They serve as a vehicle to study distributed interactions across multiple devices and ubiquitous computing environments in the wild.\\

The AtomicOrchid (chapter \ref{ch:approach}) is a serious mixed-reality game designed to mirror aspects of real-world disaster operations. In this game, field responders use smartphones to coordinate, via text messaging, GPS, and maps, with headquarters players and each other. The players in the game faces a distributed task planning problem with both time and spatial constraints. To achieve game objectives, the players need to dynamically change their team configurations. The task planning process in the game is supported by a planning support agent software. The planning support agent is based on a state-of-art coalition formation optimisation technology. Design and implementation of AtomicOrchid will be introduced in more details in chapter \ref{ch:approach}.\\

In order to explore the socio-technical issues in automated planning support systems, three studies are conducted with different research focuses. In the first study, field responders and incident commander coordinate without support of the intelligent planner. The study establish baseline performance of the game play and derived several requirements for planning support system. In the second and third studies, an automated planner was introduced to support task planning with two different interaction designs. The second study adopts the human ``On-the-loop'' design pattern in which the planning agent automatically generate plans and instruct field players to execute plans. In the third study, we adopts the human ``In-the-loop'' design in which every plan generated by the planning agent will need to be approved and edited before it is sent to field players before execution. More details of these two interaction patterns will be introduced in chapter \ref{ch:approach}.\\

The work also adopted an ethnographically-inspired approach for data analysis. We recorded both system logs and video of interaction in the field trials for analysis. To capture the distributed, concurrent nature of the interaction, four researchers with camcorders shadowed the field player teams. Qualitative interaction analysis were carried out on the collected data to unpack socio-technical issues embedded in human agent interaction.\\


\section{Scoping}\label{sec:custom}
This thesis is relevant to several research areas. \\

\begin{itemize} 

  \item Human Computer Interaction (HCI) for Disaster Response(DR). HCI is the overarching research area of the PhD work. The two sub domains of HCI is particular relevant to this thesis. This thesis follows a long-standing tradition to of empirical CSCW study, which investigates complex settings collaborative work setting and identify implications for technology support. On the other hand, the research of automation design provides framework models and terminologies to design interaction for field trials.
  \item Information and Communication Technologies in DR. With the vision of HACs system, the current ICT for DR may eventually evolve into HACs in the future. The thesis is aimed to help realise the vision by providing design implications for ICT systems with intelligent task planning agents. 
  \item Multi-agent systems (MAS). The muli-agent simulation technologies underpin the technical possibility of intelligent planning support, providing the opportunity space of human agent collective planning. 
  \item Ethnography. The observational method employed in the field studies (chapters \ref{ch:studyone},\ref{ch:studytwo},\ref{ch:studythree}) is inspired by ethnography.
%\item Human agent interaction. Existing human agent interaction research is the overarching research area of this thesis.
\end{itemize}

There are various ICT and MAS technologies designed to support disaster management activities in the different stages of the crisis circle including preparedness, response and recovery. This thesis is going to limit the scope on operations of rescue and evacuation in the immediate aftermath of a disaster impact, which typically requires high level of team coordination and real-time task planning and execution.\\ 

The thesis also focuses on socio-technical issues related to human team interacting with the automated planning support from a HCI/CSCW perspective. Although this work involves planning support agents based on multi-agent coordination algorithms, the effectiveness, performance and other technical issues of particular coordination algorithms are not concern of this work.\\

As part of ORCHID project, the AtomicOrhid serious game platform was developed as a research "Probe" to trial human agent collective planning in the domain of disaster response. The AtomicOrchid platform consists of two major components (see \ref{ch:approach} for details): a game engine, and a embeded task planning agent. The core game engine was developed , deployed and maintained by the author, whereas the task planning agent was developed by ORCHID research partners - Feng Wu and Savapali Ramchun. Both Feng and Ramchun have expertise and research interest in the performance of task planning algorithm, while the author's research interest is the interaction between human and the intelligent task planning agent. \\

\section{Research quetions}
The recent advances in ICT and multi-agent optimisation technologies have created the opportunity space for automated planning support system in disaster response domain. Before we deploy such a planning support system, a deep understanding of socio-technical issues are required for appropriate interaction design between human teams and computational agents. This work adopted serious game approach to explore the interaction design space. Integrated with automated planning support, the AtomicOrchid game platform is used as a testbed for human agent interaction designs.  The AtomicOrchid is further configured with two different interaction patterns to produce game "probes" for field trials. Through field observation, this work is aimed answer the following two research questions:

\begin{enumerate}
\item[A] What socio-technical issues will emerge if we try to automate planning process in a disaster response team? socio-technical issues can be seen as an term for the issues relating socio-technical gap often found in the CSCW systems (see section \ref{sec:sociotech}), which can range from social, organisational to other interface design issues. This work is aimed to conduct an exploration of the socio-technical issues in the interaction design space of agent planning support.

\item[B] How can we design interaction to support human agent collaboration in task planning?
Following the first question, the emerged socio-technical issues will need to be handled with approprate interaction design. This work seeks to produce interaction design implications through field observation and interaction analysis, both of which are grounded in literatures(Chapter \ref{ch:methodology})
\end{enumerate}

Both two questions will be answered with respect to the two different interaction patterns, namely human ``On the loop'' and human ``In the loop'', both of which will be detailed in section \ref{sec:approachPatterns}. 

\section{Contributions} 
\begin{figure}[h]
  \centering
  \includegraphics[scale=0.5]{img/introduction/contributions.png}
  \caption{contributions}
  \label{fig:contributions}
\end{figure}


This thesis contributes to the knowledge in the following areas (figure \ref{fig:contributions}): \\
\begin{enumerate}
  \item[A] A real-world interactive prototype (i.e. AtomicOrchid) and trials to investigate team coordination in a disaster response settings.
  \item[B] The field observation of serious game trials leads to enriched understanding of socio-technical issues surrounding automated planning support in the complex collaborative work setting of DR domain.
  \item[C] For each study, field observations are further analysed to generate design implication which contribute to future deployment of automated planning support system. 
\end{enumerate}


\section{Publications of this thesis} 
Parts of the contents of this thesis have been accepted by peer-review for publication in journal and conference proceedings in the field of HCI and multi-agent system or are in submission. The core contributions include: \\


\begin{enumerate}
\item The chapters \ref{ch:approach}, \ref{ch:methodology}  present approach and methodology employed to study socio-technical issues of planning support system. Some of the ideas of these chapters expands on the contents in:\texttt{Fischer, Joel E., \textbf{Wenchao Jiang}, and Stuart Moran. AtomicOrchid: a mixed reality game to investigate coordination in disaster response." In Entertainment Computing-ICEC 2012, pp. 572-577. Springer Berlin Heidelberg, 2012.}\\

\item The exploration of requirements for building coordination support system in chapter \ref{ch:studyone}  has been published in:\texttt{ Fischer, J.E., \textbf{Jiang, W.}, Kerne, A., Greenhalgh, C., Ramchurn, S.D., Reece, S., Pantidi, N. and Rodden, T. (2014). Supporting Team Coordination on the Ground: Requirements from a Mixed Reality Game. To appear in: Proc. 11th Int. Conference on the Design of Cooperative Systems (COOP 14). Springer.}\\


\item The exploration of socio-technical issues related to human ``On the loop'' pattern reported in chapter \ref{ch:studytwo} have been published in:\texttt{ \textbf{Jiang, W.}, Fischer, J.E., Greenhalgh, C., Ramchurn, S.D., Wu, F., Jennings, N.R. and Rodden, T. (2014). Social Implications of Agent-based Planning Support for Human Teams.  In: Proc. of the 2014 Int. Conference on Collaboration Technologies and Systems (CTS 14). IEEE.}

\item The exploration of socio-technical issues related to human ``In the loop'' pattern reported chapter \ref{ch:studythree} have been submitted as:\\



\end{enumerate}

Other contributions include:

\begin{enumerate}
\item Some results from studies in chapter \ref{ch:studyone} and \ref{ch:studytwo} also appeals in the journal article:
\texttt{ Ramchurn, S. D., Wu, F., Fischer, J. E., Reece, S., \textbf{Jiang, W.}, and Roberts, S. J., et al. (2015). Human-agent collaboration for disaster response. Journal of Autonomous Agents and Multi-Agent Systems.}\\

\item The game probe `AtomicOrchid' built in this PhD work is a central component of HAC-ER (Human Agent Collectives for Emergency Response) system  developed as main demonstrator of ORCHID project (orchid.ac.uk). The demonstrator is presented in the paper:\texttt{  Ramchurn, S. D., Simpson, E., Fischer, J. E., Huynh, D. T., Ikuno, Y., Reece, S., and \textbf{ Jiang, W.} et al. (2015). HAC-ER: A disaster response system based on human-agent collectives. In AAMAS-15 : 14th Int. Conf. on Autonomous Agents and Multi-Agent Systems.} \\ 

\item The planner agent integrated in `AtomicOrchid' is based on a novel multi-agent coordination algorithm. Details of the planner agent is presented in a technical paper: \texttt{ Wu, F., Ramchurn, S. D., \textbf{Jiang, W.}, Fischer, J. E., Rodden, T., and Jennings, N. R. (2015). Agile Planning for Real-World Disaster Response. In International Joint Conference on Artifical Intelligence.}

\end{enumerate} 

\section{Structure of the Thesis}
This thesis is structured as four parts. Part I surveys the relevant background literatures. The chapter of literature review (\ref{ch:literatures}) will firstly give an overview of task planning activities together with command and control structure of DR teams, followed by a review of empirical studies related to command and control work setting. This chapter \ref{ch:literatures} then examines the state-of-art technology practices from 3 perspectives including planning support systems, ICT support, and application of Artificial Intelligence in DR. The relationship between technological support and human operators will also be examined by reviewing relevant literatures of CSCW systems, Automation design and Human agent interactions. The rest of the chapter give an overview of serious mixed reality games which underpins the foundation of research approach of this PhD work.\\

Part II develops the approach and methodology employed to study socio-technical issues of planning support system in two chapters. The first chapter develops framework of interaction patterns under which socio-technical issues can be explored. The rest of this chapter will introduce serious mixed reality game as approach to study interactions  ,followed by detailed description of a game used as testbed for this study - AtomicOrchid. Chapter 5 describes the methodology used to study the interactional issues. In particular, this chapter will describe ethnographic observation and interaction analysis, which is supplemented by interviews and questionnaires. \\ 

Part III covers observational studies in this thesis. Chapter \ref{ch:studyone} reports the first observational study with AtomicOrchid. This version of AtomicOrchid do not have planning support agent included. The study establishes baseline human performance of task planning and derives general requirements of communication support. The chapter \ref{ch:studytwo} give an account of second observational study of AtomicOrchid. In this study, an planning agent was built into the game with human-on-the-loop interactional arrangement. Chaper \ref{ch:studythree} reports the third field study of AtomicOrchid with human-in-the-loop arrangement. \\ 

The Part IV concludes this thesis with a summary of discoveries, contributions, limitations and future work.\\









