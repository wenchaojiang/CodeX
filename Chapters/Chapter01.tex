%************************************************
\chapter{Introduction}\label{ch:introduction}
%************************************************
This Chapter will give an overview of the PhD work including research objectives, approach, research questions and contributions, followed by a list of publications related to this thesis and an overview of thesis structure.


\section{Problem Definition and Objectives}
Disaster response operations such as Urban Search And Rescue (USAR) can be very challenging. In large-scale disaster, DR teams may have limited resource and personnels to deal with multiple incidents across a large impact area. Task planning and execution need to be carried out by geographically distributed DR teams in real time against uncertainties in the environment. The challenge requires DR teams to carry out highly coordinated activities in an uncertain task environment. The challenges create the opportunity of technology support for real time Task planning and execution.  \\ 

Recently, various Information and Communication Technologies (ICT), ranging from communication infrastructures to social media platforms, have been playing increasingly significant role in the disaster management.  Moreover, Muti-agent system researchers have devised various real-time task planning algorithms to automate planning in time critical task domains such as disaster response. The advances in both ICT and Multi-agent algorithms lead to the opportunity of intelligent planning support in the DR domain.\\

However, many CSCW literatures have pointed out ill-designed work-flow management/automation system can lead to undesirable results, not only fail to improve work efficiency but also hinders human performance. Field studies of CSCW technologies have shown that it is vital to study technology in use to understand potential tensions raised for teamwork. Bowers et al. found that extreme difficulties might be encountered when introducing new technology support for human teams. New technologies might not support, but may disrupt smooth workflow if they are designed in an organisationally unacceptable way. \\

We believe the same is true for intelligent planning support. Before we can build intelligent systems that support human task planning, field trials are needed to understand the potential impact of technology support for team coordination. Although most multi-agent coordination algorithms have been tested to perform well in the computational simulations, they have never been excised to guide real human responders in the real world environment. Currently, there are few studies aimed to unpack the interactional issues relating to a socio-technical aspects of the intelligent planning support system. Interactional issues of a human-agent system can be defined as the issues related to interaction design and more importantly, the social aspects of the a human-agent. system. "Social issues" of human-agent systems are thought to be as important as technical issues []. Therefore, this PhD work is aimed to fill this gap by exploring and unpacking interactional issues of intelligent planning support system from a HCI perspective.\\

\section{Approach}\label{sec:custom}

To meet our research objective, we adopt a serious-mixed reality games approach (Fischer et al., 2012) to create a game probe (i.e. AtomicOrchid) that enables studying team interaction with planning support system in a real-world disaster scenario whilst providing confidence in the efficacy of behavioural observations. Mixed-reality games bridge the physical and the digital (Benford et al., 2005). Arguably, They serve as a vehicle to study distributed interactions across multiple devices and ubiquitous computing environments in the wild (Crabtree et al., 2006).\\

The AtomicOrchid is a serious mixed-reality game designed to mirror aspects of real-world disaster. In this game, field responders use smartphones to coordinate, via text messaging, GPS, and maps, with headquarters players and each other. The players in the game faces a distributed task planning problem with both time and spatial constraints. To achieve game objectives, the players need to dynamically change their team configurations. The task planning process in the game is supported by a planning support agent software. In Chapter x, design and implementation of AtomicOrchid will be introduced in more details.\\

In order to explore human agent interactional issues, three studies are conducted with different research focuses. In the 1st study,field responders and incident commander coordinate without support of the intelligent planner. The study establish baseline performance of the game play and derived several requirements for planning support system. In the second and third studies, an intelligent planner was introduced to support task planning with two different interactional arrangement. The second study adopts a arrangement of human-on-the-loop in which the planning agent automatically generate plans and instruct field players to execute plans. In the third study, we adopts an human-in-the-loop arrangement in which every plan generated by planning agent will need to be approved and edited before it is sent to field players for execution. More details of these two interactional arrangements will be introduced in Chapter x.

The work also adopted an ethnographically-inspired approach for data analysis. Game plays were recorded and qualitative interaction analysis were carried out to unpack human agent interactional issues.\\


\section{Scoping}\label{sec:custom}

This thesis is relevant to several big research areas. \\

\begin{itemize}
  \item Information and Communication Technology (ICT) for Disaster Response(DR). With the vision of HACs system, the current ICT for DR may eventually evolve into HACs in the future. The thesis is aimed to help realise the vision by providing design implications for ICT systems with intelligent task planning agents. 
  \item Multi-agent systems. The muli-agent simulation technologies form the technical base of intelligent planning support, providing the opportunity space of human agent collective planning. 
  \item Human agent interaction. Existing human agent interaction research is the overarching research area of this thesis. 
  \item Ethnography. The thesis adopts a Ethnographically-inspired approach to study human system interactions in field trials.
\end{itemize}

There are various ICT and AI-based technologies supporting disaster management activities in the different stages of the crisis circle including preparedness, response and recovery. This thesis is going to limit the scope on operations of rescue and evacuation in the immediate aftermath of a disaster impact, which typically requires high level of real time task planning and execution.\\ 

The thesis also focus on socio-technical issues related to human team interacting with the intelligent planning support from a HCI perspective. The work involves planning support agents based on multi-agent coordination algorithms, but the effectiveness of particular coordination algorithms are not concern of this work.\\

This PhD work is sponsored by ORCHID project (EPSRC grant xxxxxxx).The work is closely related to disaster response vignette the of the ORCHID project. As computation increasingly pervades the world around us, The ORCHID project envisions a future in which human and computational agents operating together in a large global scale, forming human agent collectives (HACs). The objective of ORCHID project is to realise the vision by studying the principled science that allows us to reason about the computational and human aspects of these systems (www.orchid.ac.uk).\\ 

As part of ORCHID project, the AtomicOrhid serious game platform was developed as A research "Probe" to trial human agent collective planning in the domain of disaster response. As mentioned in section x.x,  the AtomicOrchid platform consists of two major components: a game engine, and a embeded task planning agent. The core game engine was developed , deployed and maintained by the author, whereas the task planning agent was developed by ORCHID research partners - Feng Wu and Savapali Ramchun. Both Feng and Ramchun have expertise and research interest in the performance of task planning algorithm, where the author's research interest is the interaction between human and the intelligent task planning agent. 

\section{Research quetions}

What interactonal issues will emerge if we try to automate planning process 

How can we design interaction to support human agent collaboration in task planning. 

\section{Contributions} 
This thesis contributes to the knowledge in the following areas: \\
\begin{enumerate}
  \item[A] The field observation of serious game trials leads to enriched understanding of interactional issues surrounding human agent interaction in disaster response domain.
  
  \item[B] For each study, field observations are further analysed to generate design implication which contribute to future deployment of agent-based planning support algorithms. 
\end{enumerate}

\section{Publications of this thesis} 
Fischer, Joel E., Wenchao Jiang, and Stuart Moran. "AtomicOrchid: a mixed reality game to investigate coordination in disaster response." In Entertainment Computing-ICEC 2012, pp. 572-577. Springer Berlin Heidelberg, 2012.\\

\noindent Fischer, J.E., Jiang, W., Kerne, A., Greenhalgh, C., Ramchurn, S.D., Reece, S., Pantidi, N. and Rodden, T. (2014). Supporting Team Coordination on the Ground: Requirements from a Mixed Reality Game. To appear in: Proc. 11th Int. Conference on the Design of Cooperative Systems (COOP 14). Springer.\\

\noindent Jiang, W., Fischer, J.E., Greenhalgh, C., Ramchurn, S.D., Wu, F., Jennings, N.R. and Rodden, T. (2014). Social Implications of Agent-based Planning Support for Human Teams. To appear in: Proc. of the 2014 Int. Conference on Collaboration Technologies and Systems (CTS 14). IEEE.\\





\section{Structure of the Thesis}







