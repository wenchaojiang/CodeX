%************************************************
\chapter{Introduction}\label{ch:introduction}
%************************************************
 


\section{Problem Definition and Objectives}
Disaster response operations such as urban search and rescue (USR) can be very challenging. In large-scale disaster, DR teams may have limited resource and personnels to deal with multiple incidents across a large impact area. Task planning and execution need to be carried out by geographically distributed DR teams in real time against uncertainties in the environment. The challenge requires DR teams to carry out highly coordinated activities in an uncertain task environment. The challenges create the opportunity of technology support for real time coordination.  \\ 

(Scoping coordintion -> task planning:  Task planning is very important aspect of DR team coordination, this work focuses on task planning)\\

Recently, various Information and Communication Technologies (ICT), ranging from communication infrastructures to social media , have been playing increasingly significant role in the disaster management.  Moreover, Muti-agent system researchers have devised various real-time coordination algorithms to support team coordination in time critical task domains such as disaster response. The advances in both ICT and Multi-agent systems lead to the opportunity of intelligent planning support in the DR domain.\\

However, many CSCW literatures have pointed out ill-designed work-flow management/automation system can lead to undesirable results, not only fail to improve work efficiency but also hinders human performance. field studies of CSCW technologies have shown that it is vital to study technology in use to understand potential tensions raised for teamwork. Bowers et al. found that extreme difficulties might be encountered when introducing new technology support for human teams. New technologies might not support, but may disrupt smooth workflow if they are designed in an organisationally unacceptable way. \\

We believe the same is true for intelligent planning support. Before we can build intelligent systems that support human team coordination, field trials are needed to understand the potential impact of technology support for team coordination. Although most multi-agent coordination algorithms have been tested to perform well in the computational simulations, they have never been excised to guide real human responders in the real world environment. Currently, there are few studies aimed to unpack the interactional issues relating to a socio-technical aspects of the intelligent planning support system. Therefore, this PhD work is aimed to fill this gap by exploring and unpacking social issues of intelligent planning support system from a HCI perspective.\\

\section{Scoping (Research area)}\label{sec:custom}

There are various ICT and AI-based technologies supporting disaster management activities in the different stages of the crisis circle including preparedness, response and recovery. This thesis is going to limit the scope on operations of rescue and evacuation in the immediate aftermath of a disaster impact, which typically requires high level of real time task planning and execution.\\ 

The thesis also focus on socio-technical issues related to human team interacting with the intelligent planning support from a HCI perspective. The work involves planning support based on multi-agent coordination algorithms, but the effectiveness of a particular coordination algorithms are not major concern of this work.\\

(Approach?)The work adopted an ethnographically-inspired approach. A serious game approach was also adopted to simulate interaction and task environment of DR operation. Game plays were recorded and interaction analysis were carried out unpack important interactional issues. \\

(Use a pie chat to illustrate related research area.) \\

(State the ORCHID project, and How my work is positioned with the project, Feng is doing the AI part and I am interested in the interactions )


\section{Goal of the study}\label{sec:custom}
(Social issues and interacgtional issue are both used, need to clarify)

"Social issues" of human-agent systems are thought to be as important as technical issues [].The primary goal of this thesis is to understand the "social issues" that might emerge from  planning agent for responder teams in DR operation. \\

Adopting Serious Game approach, we conducted several field studies to get insight into potential social issues associated with two different interactional arrangements. Interactional arrangement definite how a software agent interact with a human team. For supporting responder teams, we envision two relatively straightforward interactional arrangements which is detailed in Chapter X. With the social issues unpacked, we also aim to propose design requirements and solutions to avoid and address those social issues.\\

\section{Contributions}\label{sec:custom}

contribute to understanding of interactional issues ...

(Check contributions from two papers and put it here.)

\section{Structure of the Thesis}\label{sec:issues}







