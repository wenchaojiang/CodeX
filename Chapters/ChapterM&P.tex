\chapter{Approach}\label{ch:approach}
This PhD work adopts a socio-technical view towards the planning support systems in disaster response setting. Introducing a planning support to Disaster Response(DR) operations may create a socio-technical gap that need to be considered by system designers. We argue that the gap can be reduced by an appropriate interaction design supported by deep understanding socio-technical issues surrounding the planning support. In order to gain insight into the socio-technical issues, we adopted an ethnographic approach to explore and unpack interactions between human and planning support agent in disaster setting. A serious mixed reality game (MRG) approach is adopted to create MRG AtomicOrchid(AO), which are used to simulate DR operations. We outline two interaction designs that are later implemented in AO platform for field studies. In addition, efforts have been made to establish contact with professional response agency Rescue Global(RG), which leads a workshop centered on the AO game. Professional feedbacks on AO are collected from the discussions in the workshop. \\

In this chapter, we go through the socio-technical perspective of planning support system adopted by this PhD work (section \ref{sec:sociotech} ), followed by introduction serious mixed reality game (MRG) approach (section \ref{sec:SMRG}) and description of AtomicOrchid platform (section \ref{sec:AOdescription}). The section \ref{sec:patterns} outlines the two interaction designs that were deployed to AO for field studies, and the section \ref{sec:rg} gives an introduction of the workshop with RG for professional feedbacks. \\


\section{Adopting the socio-technical perspective}\label{sec:sociotech}
This research is aimed to inform the design of planning support for organizational work conducted by responder teams. This PhD work adopts the a socio-technical view on the responder teams and their technological supporting systems. Integrating new technology support into a human organisation is a well-known challenge for socio-technical system design. In this PhD study, we anticipate the same challenge will be encountered for  introducing the planning support to disaster response operations.\\

The term socio-technical systems are used to describe systems that involve a complex interaction between humans, machines and the environmental aspects of the work system (see section \ref{sec:LRSocialTechnical}). The term stands for the recognition of both technical  and social subsystems and the very complex relationship between the two. Social systems are characterized by phenomena such as communication and cooperation between human individuals, emergence of meaning systems, self-referential development of structures. In contrast, technical systems are characterized by artifacts, control, anticipation, state-transitions, pre-programmed adaptability, learning in respect to purposes which are determined from outside the system[]. Introducing a technology support system into an organization requires the technical system and a social aspects to be integrated.\\

In the context of this study, a multi agent coordination algorithm is used to build a automated planning agent. The agent requires a set of rigid inputs and produces task assignments. On the other hand, planning activities of responder teams are characterized by natural social processes such as communication, negotiation and cooperation. To support the responder teams with multi agent coordination technologies, the confrontation between social and technical is inevitable. \\

Researchers have pointed out the existence of the inherent social technical gap - the great divide between what we know we must support socially and what we can support technically.  Some argued that human activity is highly flexible, nuanced, and contextualized and that computational entities and processes such as information sharing, roles, and social norms need to be similarly flexible, nuanced, and contextualized. However, current technology support systems for organisations are often rigid and inflexible, failing to fully support the social world. Computer science is learning how to use techniques such as machine learning, user modelling to fill social-technical gap (section \ref{sec:LRHAI}). It is hard to disprove that a technical solution is imminent. However, some argued that such a technical solution is unlikely, given that computer science, artificial intelligence (AI), information technology, and information science researchers have attempted to bridge the gap without success for at least 20 years \cite{Ackerman2000}. \\

We argue that the social aspect does not need to be fully supported through technological advance. A deep understanding of social issues and appropriate interaction design may lead to possible ``workaround'' of the socio-technical gap. Although technology support may be not fully integrated with the social aspects, we believe appropriate interaction design can reduce its negative impact on social process so that the benefits of technology can overweight its adverse social impact. \\

%The field of HCI achieved widespread recognition with its inherent focus on the importance of the interaction between people and technology at the fundamental level rather than just the design of the user interface. It explicitly recognised the importance of the roles of the social and technical aspects of work. HCI Literatures have identified a wide range of different methodologies that helps inform the development of socio-technical systems such as (1) Cognitive Work Analysis, (2) Ethnographic workplace analysis, (3) Human-centred design. Some of the recognised HCI methodologies underpin the serious game approach employed by this PhD work to understand socio-technical design space of agent-based planning support. [The switch to HCI is not smooth]\\

%[Field trials supported by interaction analysis  speech act]\\

%[Andy s literature review of 2004 supporting ethnographic study should be cited]\\


\section{Serious Mixed Reality Game as a testbed} \label{sec:SMRG}
One of our work`s main objectives is to study interaction and coordination situated in rich and `messy' real-world socio-technical settings. As it is difficult to deploy technological prototypes in real disasters, serious game approach has been adopted by researchers to study technology interaction in disaster scenarios through game-like simulations, for example to prepare first responders for scenarios in which hazardous materials are involved \cite{Losh2007}. \cite{Abbasi2012} presented a study in which locally distributed participants played the role of victims asking for help via social media in a simulated crisis, and participants that played the role of first responders used a coordination system to filter messages and mobilize the appropriate responder teams according to their assigned capabilities.\\

This PhD work also adopts serious game approach to simulate a disaster response setting in which distributed responder teams coordinate under time and spatial constraints (see section \ref{sec:LRtaskplanning}). More specifically, we create a Mixed Reality Game (MRG) as a testbed that enables studying team coordination, interaction and communication in a real-world disaster scenario whilst providing confidence in the efficacy of behavioural observations. The Mixed-reality games are recreational experiences that make use of pervasive technologies such as smart phones, wireless technologies and sensors with the aim of blending game events into a real world environment. The MRGs serve as a vehicle to study distributed interactions across multiple devices and ubiquitous computing environments `in the wild' (see section \ref{sec:LRMRgame}).\\

The MRG testbed called AtomicOrchid (AO) simulates a radioactive incident. Participants of the game play both the role of responders `on the ground', and coordinators in the control romm. They coordinate with each other through GPS, map sharing and messaging, to achieve game objectives. The AO game system can be integrated with planning agents to support players on the ground, and the interaction layer between players and agents can be configured in different ways through modifications on the game interface. Through agent integration and interface modifications, we created three `probes' of agent planning support with different interaction designs. The three probes are then be used to conduct behavioural studies, which allows us to unpack human-agent interaction with different interaction designs
.\\

\section{The AtomicOrchid Platform}\label{sec:AOdescription}
We designed and implemented the mixed reality game AtomicOrchid(AO) as a testbed for our field trials of different system interaction designs. The game involves field players on the ground (play as field responder) and online players in a control room (play as Headquarter (HQ)).

\begin{figure}[h]
  \centering
  \includegraphics[width=1\textwidth]{img/approach/GameComponents}
  \caption{HQ and field players in AO}
  \label{fig:AOroles}
\end{figure}

In the following sections, we give a detailed description of the game design, which covers  grounding of the design rationale,  iterative design process, and the system architecture.

\subsection{Game Design Rationale} \label{sec:gameRatinale}
The AtomicOrchid is based on the fictitious scenario of radioactive explosions creating expanding and moving radioactive clouds that pose a threat to responders on the ground (field responders), and the (virtual) targets to be rescued from around the game area. We chose a radiation scenario because other than disasters that cause physical devastation it poses an invisible threat, which creates the need to monitor the environment closely with sensing devices, and communicate frequently.\\

Field responders are supported by a centrally located headquarters (HQ) control room, staffed by coordinators who exchange messages with field players through an instant messaging style communication system. The messages are broadcasted, which means they are visible to all players. Core game mechanics are designed to allow us to explore specific aspects of team coordination. In particular, this is inspired by the real coordination challenge of resource and task allocation to coordinate spatially distributed resources and personnel.\\

%Whilst formal response teams tend to use radio to communicate (e.g., Toups et al., 2011) we chose text-based messages for its flexibility to support scenarios with many distributed (volunteer) field responders.\\



The game`s two-tiered organisational structure is derived from real world disaster response organisation and from NIMS (Homeland Security, 2008). The game`s HQ is loosely modelled on sector coordinators, whose role is to manage resources and communications between their assigned teams, and command and coordinate action within their sector (INSARAG, 2012). Field responders are modelled on team leaders and members. We ignore this distinction to simplify roles, assignments, and game mechanics.\\

\textbf{Responder roles and targets}. Each field responder is assigned one of four roles:\\

\begin{figure}[h]
  \centering
  \includegraphics[width=1\textwidth]{img/approach/AOroles}
  \caption{The AO targets}
  \label{fig:AOroles}
\end{figure}

There are four types of (virtual) targets:\\

\begin{figure}[h]
  \centering
  \includegraphics[width=1\textwidth]{img/approach/AOtargets}
  \caption{The AO targets}
  \label{fig:AOtargets}
\end{figure}

The objective of the field responders is to rescue as many targets as possible by `carrying' them to a drop off zone. To pick up and carry one of the target objects, two responders with particular appropriate roles are required in immediate proximity to the object. For example, a soldier and a transporter are required to pick up and carry fuel, and a medic and a soldier are required to pick up an animal (see figure \ref{fig:roleTargetMapping}).\\

\begin{figure}[h]
  \centering
  \includegraphics[width=0.3\textwidth]{img/approach/roleTargetMapping}
  \caption{Role target mapping}
  \label{fig:roleTargetMapping}
\end{figure}

The role-target mapping mechanic requires players to engage in resource coordination. Field responders have to engage in `agile teaming' forming, disbanding, relocating and re-forming in teams over the course of the game in order to complete the game objective. This is an example of what Toups et al call, information distribution (2011).\\

\textbf{The radioactive cloud}. The cloud is a danger zone that can incapacitate field responders. It imposes spatial and temporal constraints on task performance and well- being. The cloud is analogous to various spatial phenomena in disasters (e.g. spreading fires, diseases and floods). In require communication between HQ and field responders, the spatial position and movement of the cloud is only known to HQ. The cloud is shown in a heatmap style in the figure \ref{fig:cloud}. \\

\begin{figure}[h]
  \centering
  \includegraphics[width=0.8\textwidth]{img/approach/radioactiveCloud}
  \caption{The radioactive cloud}
  \label{fig:cloud}
\end{figure}

\textbf{Command-and-control structure}. The division of responsibility into HQ and field responders simulates a situation where volunteer responders are connected to a simple two level Command-and-control structure, similar to the real-time layer of the existing professional disaster response organizations (e.g., Chen et al., 2005).\\

\textbf{System interface}.System interface design is closely related to specific interaction designs, and it keeps evolving throughout three iterations of field trials. Therefore, the details of interface evolution is left to be introduced in the subsequent chapters (Chapter \ref{ch:studyone},\ref{ch:studytwo},\ref{ch:studythree}) of field trials. \\



\subsection{System Architecture}
The AtomicOrchid is based on the open-sourced geo-fencing game MapAttack that has been iteratively developed for a responsive, (relatively) scalable experience. Our mixed-reality game relies especially on real-time data streaming between client and server. The client-server architecture is depicted in figure \ref{fig:sysArchitecture}. Client-side requests for less dynamic content use HTTP. Frequent events, such as location updates and radiation exposure, are streamed to clients to avoid the overhead of HTTP. In this way, field responders are kept informed in near real-time.\\

\begin{figure}[h]
  \centering
  \includegraphics[width=1\textwidth]{img/approach/systemArchitecture}
  \caption{System Architecture}
  \label{fig:sysArchitecture}
\end{figure}

The platform is built using the Sinatra for Ruby, and state-of-the-art web technologies such as socket.io, node.js, redis and Synchrony for Sinatra, and the Google Maps API. Open source mobile client apps exist for iPhone and Android; we adapted an Android app to build the Mobile Responder App.\\


\subsection{The planning agent}
In study 2 and 3 (chapters \ref{ch:studytwo},\ref{ch:studythree}), planning agents are integrated into the AtomicOrchid to support player's planning activities. Two types of agent has be used in study 2 and 3 respectively. In what follows, we briefly describe technical details of the agent and system integration between AO and the agents.  \\

The coordination problem (described in section \ref{sec:gameRatinale}) is modelled using a Multi-Agent Markov Decision Process (MMDP) that captures the uncertainties of task execution, extending earlier work []. The modelling allows responder actions to be delayed or to fail during the rescue process. The MMDP modelling leads to a large search space, even with a small-sized problem. Hence, we devised an approximate solution to save computation time, which can be executed to support real time planning. The planning algorithm takes into account both time (cloud and human movement speed) and spatial (path planning for responders) constraints. The planning algorithm run by the planning agent produces high task allocations that minimise the travelling distance of first responders, and maximise the number of targets rescued. Before the agent was deployed to support human teams in the game setting, computational simulations were used to benchmark our MMDP algorithm against greedy and myopic methods (see figure \ref{fig:agentBenchmarking}). The results confirm that our algorithm produces efficient task allocations. It should the agents are developed by ORCHID Reseach partners from Southampton, more technical details of the planning agent is available in [JAMASS paper].\\

\begin{figure}[h]
  \centering
  \includegraphics[width=1\textwidth]{img/approach/agentBenchmarking}
  \caption{Result for MMDP, Myopic and Greedy algorithms}
  \label{fig:agentBenchmarking}
\end{figure}

For integration, the agent is deployed on a separate server. It communicate with the AO game server through a pre-defined HTTP protocol (for details, see Appendix X). The agent takes game status from game server as input, which includes player's health, road connectivity, locations of players, targets and radioactive clouds. The output of the agent a set of task assignments likes `player A and player B, go to target C' (see figure \ref{fig:inputoutput}) . The task assignments are sent to the AO game server and present to game players. Detailed interaction design between human and the agents will be presented in Chapter \ref{ch:studytwo},\ref{ch:studythree}. In order to facilitate the different interaction designs, the input of the agents are slightly different between study 2 and study 3, which will be detailed in section \ref{sec:studyoneagent} , \ref{sec:studytwoagent}. \\

\begin{figure}[h]
  \centering
  \includegraphics[width=0.5\textwidth]{img/approach/inputoutput}
  \caption{Input and output of the agents}
  \label{fig:inputoutput}
\end{figure}

\subsection{Log system}

\subsection{Iterative design and development}
Before the game is deployed for observational studies in this PhD work, the game went through a iterative design and development process to test, refine game concepts and system robustness. We briefly describe three cycles of iterative game design and evaluation before the system is ready for the first formal field study.\\

In the first iteration, we used a paper-based prototype to test and refine the core game mechanics. We recruited 12 participants, allocated one of four roles to them, and equipped them only with paper maps with locations of targets. They had to form different kinds of teams to retrieve the different kinds of boxes placed in the game area. The paper prototype demonstrated the demand for better support of situation awareness and communication to enable coordination.\\

The technology prototype was first tested with users in the second iteration. Users were equipped with the responder smartphone app to communicate, navi- gate, locate and pick up targets in teams formed according to role requirements. HQ was staffed by members of the research team. A pilot study was conducted with members of the public that visited an Open Day at a local university. A total of 20 members of the public tested the game in four ad-hoc game trials. The les- sons learned in the pilot study revealed problems with user interaction, network- ing, and game parameter tuning, which we subsequently addressed.\\

In the third iteration, we improved system stability and interface designs. We conducted a pilot study at the campus of another university, to test the system in place. The full-fledged study we report on here was conducted shortly thereafter.\\


\section{Explore interaction designs with three AtomicOrchid studies}\label{sec:patterns} \label{sec:approachPatterns}

[Justify the relation between the three iterations]
- avoiding pitfalls that undermines observation of interaction arrangement
- HQ agent interaction can be parallel/ HQ FR interaction follows progressive design interaction
- The first iteraction: base case/ need to understand the organization before we do anything.\\

Based on serious game approach, three studies are planned to explore the interactional issues related to the socio-technical integration of the planning agents and the responder team. To build such a socio-technical system, there are various ways to arrange the interaction between responder teams and a planning support agent. Inspired by the model of Level Of Automation (LOA) from the research of automation design, we outlined 4 paradigms of human agent interaction loosely based the automation level of planning activities: Full manual, Human-in-the-loop, Human-on-the-loop, and Human-out-of-loop. Arguably, the paradigm of Human Out-of-loop is believed to be unrealistic compared to In-the-loop, On-the-loop and full manual.Therefore, this PhD work will only consider the letter 3 notions of automation.\\

In research of automation design, the LOA model has been developed to categorise systems into a linear spectrum according to degree of automation (reference to literature review). Arguably, the model may not fit into context of socio-technical system due to some of its limitations identified in section x.x.x. However, the terminologies that come with the model can still serve as a reference point for interaction designs to be studied in this PhD work.

%There are several variations of the LOA models. One example is the two-dimensional LOA model proposed by Parasuraman (detailed in literature review section x).Several limitations of the LOA has been identified in chapter x. Most importantly, the LOA model has a focus on one-to-one operator-system interaction (e.g. autopilots, tele-oportation). It fails to capture complexity of the interactions in the context of technology support for organisational work. Therefore, it does not fit into the context of socio-technical system, which is the main focus of this research. 

\begin{enumerate}
\item Human Out-of-the-loop
Out-of-the-loop represents the highest level of automation. Out-of-the-loop system is supposed completely run independently. Human is replaced with machine, therefore no human system interaction is required. It is unlikely to be realized in a socio-technical system in which organisational work is mainly carried out by human and supported by technologies.  \\

\item Human On-the-loop 
In this research, we use the term Human On-the-loop to describe a system with high level of automation, which requires minimum level of human intervention. Compared to Out-of-the-loop, the On-the-loop system is designed run without human intervention at most of the times. However, human supervision and intervention are still required for contingencies. 

\item Human In-the-loop 
In this research, Human In-the-loop represents  a system with medium level of automation. Compared to the On-the-loop system, the In-The-Loop system can not run without human input. Constant human interactions are required to achieve goal of the socio-technical system. \\

\item Full manual 
In full manual system describe a system without automation. In the context AtomicOrchid, the platform without integration of planning agent  can be seen as a full manual system. 

\end{enumerate}

In the context of AtomicOrchid platform, the notions of In-the-loop and On-the-loop can be used to describe the degree to which the planning agent automate the real-time task planning and to what extent human Headquarters need to be involved in the plan-execution loop. Guided by the 2 notions, we devised two detailed interaction designs for integrating the planning agent into AO game. In next the two sections, we will detail the two interaction designs, followed up by an overview of three field studies, which details how a series of system prototyping  and field trials are organised based on the two interaction designs, and how they are designed to serve the research objectives.\\

\subsection{The On-the-loop interaction}
The On-the-loop interaction is designed to facilitate the division of labour between humans and agent: a planning agent routinely assigns tasks to distributed responder teams, while human coordinators (the HQ) monitor and support the task execution by responding to arising contingencies (see figure \ref{fig:OnTheLoop}). In this design, the agent can directly contact field responders to allocation tasks. The responsibility of the planning agent is to generate and send plans directly to field responders. The agent is also responsible for initiating re-planning according to the changes of game status. The agent can also directly handles feedbacks from the agent. i.e. the field players can feedback to the agent by accepting or rejecting the plan, while the agent can generate new plans according to the feedbacks. \\

The role of the HQ is to monitor the planning process and provide support when contingency rises. For example, the HQ may decide to stop some tasks issued by the agent if threat of radiation increases unexpectedly. It should be noted that The agent can operate without HQ input , and the HQ intervention is supposed to be only occasional. \\

\begin{figure}[h]
  \centering
  \includegraphics[width=0.5\textwidth]{img/approach/OnTheLoop}
  \caption{On-the-loop interaction design}
  \label{fig:OnTheLoop}
\end{figure}

\subsection{The In-the-loop interaction}
The In-the-loop interaction is designed to facilitate a different pattern of labour division between humans and agent: a planning agent propose the task assignments, while the human HQ need to approve the tasks before it is sent to the field responders. In this design, the HQ can be seen as a mediator between field responder and the planning agent. If the HQ don't agree with a task allocations from agents, they can intervene by directly editing part of the plan or require the agent to re-plan. \\

The feedbacks from the field responders (i.e. accept/reject) are delivered to HQ before any actions are taken. The HQ are responsible to review the feedbacks and decide the actions to be taken (e.g. decide to initiate re-plan, or ignore). Compared to On-the-loop interaction, the agent in this design will never directly communicate with field responders and the agent can not operate without HQ's input, i.e. the HQ have to make decisions on every agent proposed tasks, and take actions on the feedbacks from field responders. \\

\begin{figure}[h]
  \centering
  \includegraphics[width=0.5\textwidth]{img/approach/InTheLoop}
  \caption{In-the-loop interaction design}
  \label{fig:InTheLoop}
\end{figure}


\subsection{The three AO game studies to explore socio-technical issue of human agent interaction}
Three studies are conducted to explore the socio-technical issues related human agent interaction. The first study focus on a ` manual' version of AtomicOrchid without agent support, while the latter two study focus on On-the-loop and In-the-loop respectively. For each study, we develop a game probe through prototyping, to facilitate the interaction design to be studied. Through field trial of the AO game probes, The studies seek to unpack human agent interactions with the three interaction designs. \\ 

The first study is aimed to observe and explore human coordination without planning agent support. The non-agent trial supports the two later (chapter \ref{ch:studytwo}, \ref{ch:studythree}) agent-supported system trials by 1) Revealing baseline performance of human coordination without agent support 2) Generate design requirements which feeds into subsequent prototyping of AtomicOrchid. The purpose of the second and third studies are aimed to investigate socio-technical issues related to the On-the-loop and In-the-loop interactions and derive design implications of interaction designs from the field observations.\\

\section{Collaboration with Professional Disaster Response Organisation}
In addition to the three observational studies of AtomicOrchid games, two workshops with a Rescue Global (a professional disaster response agency) was organised to get professional feedbacks about the AtomicOrhid system and planning support component. Because the contact with Rescue Global was established in very late stage of this PhD work, the feedbacks from Rescue Global (RG) workshop are not used to drive the development of AO simulation and interaction design, but to get an insight into similarity and difference between AO simulation and the real world disaster response(DR) operations, which help us understand limitations and strengths of our observational study.  The first RG workshop happened between study two and study three (section \ref{sec:RGworkshopone}). The In-the-loop AO probe was demonstrated to RG team and a discussion was organised to get feedbacks from RG. The second RG workshop happened after the study three (section x). It contains a hands-on session for RG to experience AO game, and the feedbacks are collected from discussions during and after the game session.\\

\subsection{Introduction of Rescue Global}\label{sec:rg}
[Need paraphrase] Rescue Global (RG) is a disaster response organisation. They are a UK charity and a US not-for-profit headquartered in London, UK. Their remit is to provide ``immediate crisis and disaster reconnaissance ability, delivering accurate and timely information and risk data, as well as performing emergency search and rescue operations where needed to save life.'' [30]. RG has adopted a framework of procedures that implements ISO 9001 Quality Management (QM) principles, which commits RG for example to conduct risk assessments and to record decisions for accountability purposes. This has implications on the ways in which missions are planned and carried out (the focus of our field work).\\

[Example of operations carried out? , give context for next feedback section below]\\

RG`s organisational structure represents a typical hierarchy found in emergency services (cf. [39]), termed Gold, Silver and Bronze. Gold denotes the strategic lead, which is associated with RG`s senior officers (often referred to as the `head shed') and the headquarters in London, Silver is the tactical lead, which is `spun up' for mission planning, both to assess feasibility of deployments and when actually deployed onsite. Bronze refers to the operational level, in which `Pathfinders' (field responders) carry out operations `on the ground' supported by Silver command [29]. RG`s core staff consists of around 20 highly specialised experts and admin support, many of whom have had prior careers in the military, and emergency and first response services.\\

%\section{A framework of interactional issues}\label{sec:interactional}
%interaction techniques 
%[Which Interaction Technique Works When] refer to interaction techniques may refer to is a set of %interface widget design.

%We follows a process of interaction design documented in the [Designing interactions p 15], the process %is interactive, with a special focus on understanding issues and generating design implications . The %issues emerges in previous iterations will be feedback to next iterations. 

%interactional issues: 

%Interface aspects: tech dependent or not?
%interaction aspects: concerned with interaction patterns 
%Social aspects: Corncerned with what? 

%(Sedig, K.Parsons, P) pattern based approach.
%(Most interaction techniques literature reviewed so far is about study in visual representation. )
%(Interaction design, wiki)

\chapter{Methodology to Investigate Human Agent Interaction}\label{ch:methodology}
This chapter takes an in-depth look at the methodology that underlies the empirical approach adopted in the presented studies.This PhD study is aimed to conduct ethnographic-oriented field studies based on AtomicOrchid(AO) platform to generate descriptive results, which contains rich interactions among participants and planning support system. Ethnographic observations and interaction analysis are central to all three field studies, while group interviews, message classification, and system log analysis are introduced to supplement the two former in-situ methods.\\

\section{Ethnomethodological perspective}
Observation of participants in the field study is informed by Ethnomethodology (EM). Following the tradition of ethnography, EM seeks to explicate real-world organisation of works by adopting the naturalistic stance. The EM places methodological emphasis on rigorous description of the situated (i.e. local, observable) actions and practices (Suchman, 1987) in and through the contingent accomplishment of daily activities. The EM-informed ethnography arguably helps answering what might be regarded as an essential question in design: what to automate and (Crabtree et al. 3) what to leave to human skill, competence, judgement, experience and expertise. By producing description of the actions and practices in and through which the work `gets done' time and time by the members, The EM could inform the system design by uncovering what actions and activities we should therefore support.\\

For the purpose of this thesis, the social situation the interaction with and around the planning support was argued to be a critical factor to understand how social organisation of work is achieved participants with the existence of a planning support system. Observation of the situated actions and practice employed by the participants was a key method for the field study. The use of the system was observed and filmed for later analysis. Video is widely recognised as an important resource for ethnographies around technology use (Crabtree et al., 2006). The next section will go through the method of video-based interaction analysis for unpacking the interactions observed in the field. \\

\section{Interaction Analysis}
Interaction analysis can be defined as an interdisciplinary method for empirical investigation of interaction of human beings with each other and with objects in their environments [Jordan and Henderson]. In the context of HCI study, it is a method of analysing naturally occurring talk and activity, with the aim of uncovering, describing something of the order and organisation by which people interpret and interact with each other and with the things around them.\\

The Advantage of Interaction analysis lies in its ability to deal with actual details of technologically mediated interactions and allows technology developers to see exactly how technology fits (or doesn't fit) into current working practice. Other methods such as questionnaires and interviews relies upon report from participants, rather than actual, reasoning and behaviour. The over-reliance on participants' report make those methods vulnerable to the problems of people producing post-hoc rationalisations of actions, forgetting or incorrectly estimating aspects of behaviour, expressing ineffective attitudes, and generally lacking insight into the tacit procedures underlying much of their activity. Instead, interaction analysis can expose the practical reasoning activities of participant's themselves in a way which does not require them having to remember, justify or even know what they did. This effectively indicates how people think and make sense of technology they are using, in the performance of some task. However, interaction analysis is extremely time-consuming, which means it can only be carried out on small number of participants. The limitation makes it unsuitable for answering to very specific design questions and for examining the needs and behaviours of diverse groups of people. Further, the generality of its findings may need to be established by other means.\\

For the purpose of this PhD work, interaction analysis is applied to the evaluation of game probes undergoing field trials in selected work settings of Disaster Response. In this case, the description generated by interaction analysis could expose information on the sequential organisation of technologically and socially mediated activities, which in turn, reveals how the activities can be supported. In this research , the main resource of interaction analysis is video recording of AO game plays. The video analysis generally consists of three stages [Heath and Luff] :

\begin{enumerate}

\item Cataloguing the data corpus: This step involves a preliminary review of the corpus. Basic aspects of the activities and events are catalogued at this stage. Preliminary reviews and cataloguing should involve no more than a simple description and classification of the materials without detailed analysis. \\

\item Selecting Episodes: In light of preliminary review of data, a more focused substantive review of data is carried out in this stage. Repeated analytical searches of the data corpus is also involved to find examples of actions that appear to reflect similar characteristics. Candidate episodes of the particular phenomena, actions or organisation under scrutiny should be gathered in put into collections. \\

\item Detailed analysis:  We begin to look more closely at the selected candidate episodes to unpack the way in which interaction is accomplished by participants. The process generally involves transcribing and analysing both talks and visible conducts in the candidate episodes. \\ 

\end{enumerate}




\section{Data collection and handling}
Field observations and interaction analysis have been introduced as the main methods for investigating socio-technical issues in our studies. This section will introduce a number of methods employed for data collection and handling. In particular, the group interview supplements field observation by providing subjective description of game play experience. The message classification method gives an quantitative insights into remote communication. It also provides context for interactions in the field and help to identify interesting game events. The log analysis produces game events visualisation and replay. When triangulated with the video data, the log data analysis also supports interaction analysis by providing context and help to identify interesting episodes in video data.\\

\subsection{Shadowing}
[Shadowing literatures] \\

[Andy] Audio recorders and video cameras are believed to valuable resources for ethnographic study. Both audio and video recordings offer us a rich resource and enable us to elucidate the methodical ways in which work is organised and accomplished as an interactional matter[]. This PhD work uses video/audio recordings to capture distributed activities in AtomicOrchid(AO) game as it happens, and the subsequent interaction analysis is based on reviewing the recordings. \\

For each AO studies, multiple researchers were hired to capture activities of distributed teams in the field. The researchers were instructed to 
follow player teams and film their actions including talking, gestures, and other bodily activities. In some cases, there wasn't enough researchers to cover all the player teams in the field. To maximise the number of teams covered, the researchers were instructed to avoid filming same player teams at a same time. In the control room, one researcher records the actions of Headquarters players with two camcorders. One camcorder was fixed on tripod and the other was held by researcher. \\

As the coverage of video recordings may be a concern, the audio recordings were used to supplement videos. An audio recording app is installed in the Android phones that are used by the field players in the AO game. The app works in the background, recording player's voices without interrupting players' use of the AtomicOrchid client app. The obvious limitation of audio is that we can not visually see player's actions with it, while its strength lays in its guaranteed coverage of all player teams at all times. As a great deal of the work of a setting is conducted through talk[], audio recordings are useful alternative resources for interaction analysis when video coverage is not sufficient. \\


\subsection{Log Data Handling} \label{sec:aprloghandling}
Now days, the HCI researchers often collect rich dataset for investigating interactions. The data set becomes larger and larger as digital record systems increase in availability [DRS]. Automated tools are increasingly necessary for managing the organisation, replaying, structured and free coding (and annotation) and analysis of these growing data sets []. For this research, the logging system of AtomicOrhid produces time-stamped system logs. The raw log data is hard to be used directly as a resource for interaction analysis. In order to reveal the information buried in the logs, the data has to be processed so that it can be easily read and triangulated with data in other modalities i.e.  video and audio recordings. There are a number of tools already in existence to automate data handling and support the analysis of interactions []. However, these tools often have limited or very specific functionalities[]. Therefore, we developed our own data visualisation and log replay system which are tailored to handle the raw log data from AtomicOrchid studies.\\

To recap (see section xxx), the logging systems records players's location, health, targets' location and status (pick up/drop off), task assignments and players' feedback (reject/accept). The data visualisation tool (see figure xxx) focus on visualising task assignments in the game play. The game events related to task allocations, including task assignments, player feedback, target pickup and dropoff, are all plotted on a time line with different annotations. The   colored dots and squares denote various game events (see figure \ref{fig:logvis}). Detailed information of the event can be displayed when mouse hover on. As you can see for figure \ref{fig:logvis}, the deep blue dot with mouse cursor hover on denotes a target pick up event. The two players with initials CY and DS (see lines connecting the pick-up events) picked up target 598 at the 10th minute of game play. This data visualisation tools give an overview of game-event sequence. It assists interaction analysis by providing context and guidance on episode selection process. \\

\begin{figure}[h]
  \centering
  \includegraphics[width=1\textwidth]{img/methodology/logVisualisation}
  \caption{Log visualisation}
  \label{fig:logvis}
\end{figure}

[Insert ]
A replay system is also built to triangulate multiple video with log data. The main map view on the replay interface (see figure \ref{fig:replay}) displays game status reconstructed from log data, in a way that similar to the HQ interface does (see section x for details). By giving a time offset to video file, the videos are synchronised with the main map view of game status. The replay system is important tool for interaction analysis, as it presents distributed game play with a single interface in a synchronised way, providing insights into the distributed interaction among participants and the system as it happened in parallels. \\

\begin{figure}[h]
  \centering
  \includegraphics[width=1\textwidth]{img/methodology/replay}
  \caption{Replay system}
  \label{fig:replay}
\end{figure}

\subsection{Message classification} \label{sec:aprmsg}
For AtomicOrchid, remote coordination between field and human HQ is achieved through a text messaging channel. The remote messages are recorded as part of system logging. To understand how the team members interact through the remote messages, we devised a message classification method based on speech act theory (Searle, 1975). We used speech-act theory and the notion of adjacency pairs in linguistics to classify messages sent between and among responders and HQ. According to speech act theory, utterances in dialogues can be considered as speech acts from three dimensions. We were primarily concerned with the illocutionary dimension of speech acts.\\

Searle`s classification of illocutionary acts (Searle, 1975) is used to categorize messages in the communication system as follows.\\

\begin{enumerate}
\item Assertives: speech acts that commit a speaker to the truth of the expressed proposition.
\item Directives: speech acts that are meant to cause the hearer to take a particular action, e.g. requests, commands and advice.
\item Commissives: speech acts that commit a speaker to some future action, e.g. promises and oaths.
\item Expressives: speech acts that express the speaker's attitudes and emotions towards the proposition, e.g. congratulations, apologies and thanks.
\item Declarations: speech acts that change the reality in accord with the proposition of the declaration, e.g. pronouncing someone guilty.
\end{enumerate}

The notion of request-response adjacency pairs are also used to gain insights into the reciprocity of communication. In linguistics, adjacency pairs describe conversational turn taking. In the Radiation Response Game, we expected many actions in remote conversation to be accomplished through pairs of utterances such as request-response, question-answer, or inform-acknowledge.\\

The purpose of message classification is to give an overview of the communication in the message channel. Meanwhile, the result of message classification supports interaction analysis, as it helps to identify interesting moments of team interactions in the game such as important decision points.\\

\subsection{Group interview}
For all three field studies, group interviews were conducted with all participants after each game sessions. The interviews consists of open-ended questions with the aim to supplementing the field observation and interaction analysis with participants' comments about their experiences of the game. The interview is `informal and unstructured' in a sense that it is not driven by a pre-defined questions, but only research scope and interest. It is conducted in the manner of a conversation taking place between the researcher and the participants [Andy].  The interview does not stand on its own and provide distinctive results. The primary aim of the interview is to development an overview of participants' experience of the game. Meanwhile, emergence of unanticipated issues and events was also fostered by asking open-ended questions, which in turn, are used to establish context of the issues for interaction analysis.\\

\section{Analytic Procedure}
To sum up, field shadowing and interaction analysis are the main in situ methods applied in the studies. Data collection and handling process are supported by methods including log data handling group interviews, and message classification. With this inventory of research methods, we will depicts a typical analytic procedure for analysing the teamwork in AtomicOrchid studies. \\

The procedure begins with field studies after which a set of data are collected from three sources including system logs, video/audio recording and group interviews. The messages logs are then classified according to speech act theory. The resulted classification gives quantitative insight into the remote communication. The further log data handling produces replay and visualisation of game events. \\

The output of data handling process and then used as resources for interaction analysis. The message classification contribute to catalogue building process in the interaction analysis by augmenting context of remote communication. The data visualisation also help us identify important episodes of interactions and provide context for further episode analysis, while the game replay triangulate video recordings with system logs. The replay system is used as the major tool for in-depth data examination in the episode analysis process, as it provides synchronised view of multiple videos and system logs. Additionally, the players' comments from the group interview give us insights into participants' subjective game experience, which are also important context for episode analysis.\\



\begin{figure}[h]
  \centering
  \includegraphics[width=1\textwidth]{img/methodology/analyticprocedure}
  \caption{Analytic procedure}
  \label{fig:analyticprocedure}
\end{figure}

